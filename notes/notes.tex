\documentclass{article}
\usepackage[top=1in, bottom=1in, left=1in, right=1in]{geometry}
\usepackage[utf8]{inputenc}
\usepackage[T1]{fontenc}
\usepackage{amsmath}
\usepackage{amssymb}
\usepackage{lmodern}
\usepackage{bm}
\usepackage{scrextend}
%\usepackage{showframe}
\usepackage{calc}
\usepackage{changepage}
\usepackage{mdframed}
\usepackage{dsfont}
\usepackage{enumitem}
\usepackage{pbox}
\usepackage{setspace}
\usepackage{stmaryrd}
\usepackage{setspace}
\usepackage[table]{xcolor}
\usepackage{multicol}
\usepackage{mathtools}
\usepackage{hanging}
\usepackage{longtable}
\usepackage{aligned-overset}
\usepackage{ntheorem}
\usepackage{mdframed}
\usepackage[theorems,breakable]{tcolorbox}%

\allowdisplaybreaks
\hbadness=10000
\makeatletter
\renewcommand*\env@matrix[1][*\c@MaxMatrixCols c]{%
  \hskip -\arraycolsep
  \let\@ifnextchar\new@ifnextchar
  \array{#1}}

\newcommand{\mathleft}{\@fleqntrue\@mathmargin10pt}
\newcommand{\mathcenter}{\@fleqnfalse}
\makeatother

\newcommand*{\QED}{\hfill\ensuremath{\square}} 
\newcommand{\R}{\mathbb{R}}
\newcommand{\N}{\mathbb{N}}
\newcommand{\Z}{\mathbb{Z}}
\newcommand{\Q}{\mathbb{Q}}
\newcommand{\p}{\mathcal{P}}
\newcommand{\K}{\mathbb{K}}
\newcommand{\C}{\mathbb{C}}
\newcommand{\la}{\langle}
\newcommand{\ra}{\rangle}


\newcommand{\bol}[1]{\mathbf{#1}}


\newlength{\hangwidth}
\newcommand{\newhang}[1]{\settowidth{\hangwidth}{#1}\par\hangpara{\hangwidth}{1}#1}

\newcommand{\norm}[1]{\lVert #1 \rVert}

\def\Abstand{1cm}

\newtcbtheorem{lemma}{Lemma}{%
        theorem name,%
        colframe=gray,%
        fonttitle=\bfseries,
        boxsep=5pt,
        left=5pt, right=5pt, top=5pt, bottom=5pt
    }{lem}

\newtcbtheorem{theorem}{Theorem}{
        theorem name,%
        colframe=gray,%
        fonttitle=\bfseries,
        boxsep=5pt,
        left=5pt, right=5pt, top=5pt, bottom=5pt
    }{thm}

\newtcbtheorem{definition}{Definition}{
      theorem name,%
      colframe=gray,%
      fonttitle=\bfseries,
      boxsep=5pt,
      left=5pt, right=5pt, top=5pt, bottom=5pt
  }{def}

\title{Compositional Generative Modeling}
\author{ }
\date{}


\begin{document}


\maketitle

\newlength\breite
\setlength\breite{\linewidth-4pt}
\setlength\fboxsep{0pt}
\setlength\fboxrule{0.25pt}
\setlength{\abovedisplayskip}{3mm} %Abstand OBEN zw. Text und Formel
\setlength{\belowdisplayskip}{3mm} %Abstand OBEN zw. Text und Formel
\setlength\itemsep{0pt}
\setlength\parindent{0pt}


\mathleft

\section*{Introduction}

Generative Models have had tremendous success in recent years, to train them to generalize in complex and changing environments is challenging, which is why compositionality is a desirable property. 

What we want, is for two learned generative models(i.e. Diffusion, Flow Matching etc.), represented by their learned densities $p_{D_1}$ and $p_{D_2}$ on some datasets $D_1$ and $D_2$ to be able to sample from $$p_{comp} \propto p_{D_1} \cdot p_{D_2}$$ where the normalization constant $Z$ given by:
\[
Z = \int_{\R^d} p_{D_1}(x) \cdot p_{D_2}(x) \, dx.
\]
is not tractable in practice (if we assume data in $\R^d$ which is not too restrictive for now). To do this, multiple naive and more involved ideas exist, which we will try to explore in this document, focussing on toy datasets.

interesting papers https://arxiv.org/abs/2302.11552 and https://arxiv.org/abs/2502.04549

\end{document}